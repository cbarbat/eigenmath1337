\subsection{Complex numbers}
When Eigenmath starts up, it defines the symbol $i$ as $i=\sqrt{-1}$.
Other than that, there is nothing special about $i$.
It is just a regular symbol that can be redefined and used for some other purpose if need be.

Complex quantities can be entered in either rectangular or polar form.

\begin{Verbatim}[formatcom=\color{blue},samepage=true]
a+i*b
\end{Verbatim}

$\displaystyle a+ib$

\begin{Verbatim}[formatcom=\color{blue},samepage=true]
exp(i*pi/3)
\end{Verbatim}

$\displaystyle \exp(\frac{1}{3}i\pi)$

Converting to rectangular or polar coordinates causes
simplification of mixed forms.

\begin{Verbatim}[formatcom=\color{blue},samepage=true]
A = 1+i
B = sqrt(2)*exp(i*pi/4)
A-B
\end{Verbatim}

$1+i-2^{1/2}\exp(\frac{1}{4}i\pi)$

\begin{Verbatim}[formatcom=\color{blue},samepage=true]
rect(last)
\end{Verbatim}

$\displaystyle 0$

Rectangular complex quantities, when raised to a power, are multiplied out.

\begin{Verbatim}[formatcom=\color{blue},samepage=true]
(a+i*b)^2
\end{Verbatim}

$\displaystyle a^2-b^2+2iab$

When $a$ and $b$ are numerical and the power is negative, the evaluation is done as follows.
$$i
(a+ib)^{-n}
=\left[\frac{a-ib}{(a+ib)(a-ib)}\right]^n=
\left[\frac{a-ib}{a^2+b^2}\right]^n$$
Of course, this causes $i$ to be removed from the denominator.
%For $n=1$ we have
%$${1\over a+ib}={a-ib\over a^2+b^2}$$
Here are a few examples.

\begin{Verbatim}[formatcom=\color{blue},samepage=true]
1/(2-i)
\end{Verbatim}

$\displaystyle \frac{2}{5}+\frac{1}{5}i$

\begin{Verbatim}[formatcom=\color{blue},samepage=true]
(-1+3i)/(2-i)
\end{Verbatim}

$\displaystyle -1+i$

The absolute value of a complex number returns its magnitude.

\begin{Verbatim}[formatcom=\color{blue},samepage=true]
abs(3+4*i)
\end{Verbatim}

$\displaystyle 5$

Since symbols can have complex values, the absolute value
of a symbolic expression is not computed.

\begin{Verbatim}[formatcom=\color{blue},samepage=true]
abs(a+b*i)
\end{Verbatim}

$\displaystyle {\rm abs}(a+ib)$

The result is not $\sqrt{a^2+b^2}$ because that would assume that
$a$ and $b$ are real.
For example, suppose that $a=0$ and $b=i$.
Then
$$|a+ib|=|-1|=1$$
and
$$\sqrt{a^2+b^2}=\sqrt{-1}=i$$
Hence
$$|a+ib|\ne\sqrt{a^2+b^2}\quad\hbox{for some $a,b\in\mathbb C$}$$

The $mag$ function can be used instead of $abs$.
It treats symbols like $a$ and $b$ as real.

\begin{Verbatim}[formatcom=\color{blue},samepage=true]
mag(a+b*i)
\end{Verbatim}

$\displaystyle (a^2+b^2)^{1/2}$

The imaginary unit can be changed from $i$ to $j$
by defining $j=\sqrt{-1}$.

\begin{Verbatim}[formatcom=\color{blue},samepage=true]
j = sqrt(-1)
sqrt(-4)
\end{Verbatim}

$\displaystyle 2j$
